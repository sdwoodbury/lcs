\documentclass {article}
\usepackage {geometry}
\usepackage {fancyhdr}
\usepackage {amsmath , amsthm , amssymb}
\usepackage {graphicx}
\usepackage {hyperref}
\newcommand{\tab}[1]{\hspace{.2\textwidth}\rlap{#1}}
\begin {document}

\tab{$T_1$ Calculation: (by line number)} \hfill \break
let sequences a and b have length n. \hfill \break \hfill \break
line 4: $O({n}^{2})$ \hfill \break 
Proof: Array c is a double array. Its length and width are both size n. It follows that c has n * n elements. The array must be written to memory, and this time can be bounded by a constant. \hfill \break \hfill \break
line 5: $O({n}^{2})$ \hfill \break Proof: Same as line 4. \hfill \break \hfill \break
line 8-19: $\sum_{i=1}^{n} { O(1) \cdot i} = O{(1)} \cdot \frac{n(n+1)}{2}$ \hfill \break
Proof: Lines 10-19 are $O{(1)}$ The maximum number of comparisons which can be made in each iteration of the loop is 2, and the number of statements, which are not comparisons, that execute each time the loop iterates is 5. \hfill \break
Line 8 iterates a.length times. for the i-th iteration of line 8, line 9 iterates i times. This is a series. \hfill \break \hfill \break
lines 21-32:  $\sum_{i=1}^{n} { O{(1)} \cdot i} = O{(1)} \cdot \frac{n(n+1)}{2}$ \hfill \break
Proof: same as lines 8-19 \hfill \break \hfill \break
lines 34-37: $O{(1)}$ \hfill \break \hfill \break
lines 39-48: work is $\Omega {(n)}$  and $O{(2n-1)}$ \hfill \break
Proof: The path along a nXn grid is at least n. The loop starts at the bottom right corner. If the path goes only left and not up, that is a path of length n. The longest path which can be taken is to go up and left as many times as possible. The loop stops when there is either no more room to go up or no more rom to go left. So the longest path is n-1 + n. \hfill \break \hfill \break
Lines 49-50: $O{(n)}$ \hfill \break
Proof: the longest common subsequence between two sequences, of length n, is n. If it was longer, that would contradict the premise that each sequence is of length n. \hfill \break \hfill \break
line 52: $O{(1)}$. \hfill \break
Proof: The array index could be cached, or a disk read may be required. The maximum read time is bounded by the time it takes to read the element from disk. \hfill \break \hfill \break
$T_1 = O{({4n}^{2} + 5n - 1)}$ \hfill \break \hfill \break \hfill \break

\tab{$T_\infty$ Calculation: (by line number)} \hfill \break
The work for this calculation is $T_1$ for lines 1-5 and 34-52 plus the work to computer lines 8-32 in parallel. \hfill \break \hfill \break

lines 8-19: $O{(n)}$ \hfill \break
Proof: With infinite processors, the inner loop, lines 9-20, can be executed in parallel. The work for this would be $O{(1)}$. The outer loop, line 8, can not be executed in parallel. The work for this would be $O{(n)}$. \hfill \break \hfill \break
lines 21-32: $O{(n)}$ \hfill \break
Proof: same as the proof for lines 8-19 \hfill \break \hfill \break
$T_\infty = O{({2n}^{2}+2n+1)}$ \hfill \break \hfill \break

$\textbf{Parallelism: } O{(\frac{{4n}^{2}+5n-1}{{2n}^{2}+2n+1})}$
\end {document}


